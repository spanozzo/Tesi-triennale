% !TEX encoding = UTF-8
% !TEX TS-program = pdflatex
% !TEX root = ../tesi.tex

%**************************************************************
\chapter{Valutazione retrospettiva}
\label{cap:valutazione-retrospettiva}
%**************************************************************

\section{Soddisfacimento degli obiettivi}
Prima dell'inizio dell'attività di stage, io e il mio tutor abbiamo redatto il Piano di Lavoro. All'interno di esso, oltre ad una sintesi dell'organizzazione del tirocinio e ad una serie di ulteriori informazioni, abbiamo fissato anche gli obiettivi obbligatori e desiderabili che avrei dovuto raggiungere nell'arco delle 320 ore di stage, visibili in \hyperref[obiettivi_stage]{questa} tabella.\\
Gli obiettivi desiderabili che abbiamo fissato erano tutti degli incrementi rispetto a quelli obbligatori. In questo modo mi sarei potuto concentrare su quelli più importanti per primi e solo in seguito, una volta finiti, approfondire gli altri.\\ 
Sono riuscito a raggiungere tutti gli obiettivi obbligatori e parzialmente quelli desiderabili. Di quest'ultimi, infatti ho solo in parte soddisfatto il seguente:
\begin{itemize}
	\item Sviluppo di un'applicazione Java EE \textit{3-tier}, composta da un'interfaccia grafica HTML5/Angular in grado di visualizzare i dati dei risultati in modalità grafica e un Web Service RESTful per la presentazione dei dati in formato JSON recuperati con Hive/Impala.
\end{itemize}
Questo perchè, come precedentemente già descritto, zsdkòoiljndegrfaslkijmnsvfbdòlknsfdgòlknfxdbv


\section{Conoscenze acquisite}


\section{Valutazione personale}