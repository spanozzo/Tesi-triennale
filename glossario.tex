
%**************************************************************
% Acronimi
%**************************************************************
\renewcommand{\acronymname}{Acronimi e abbreviazioni}

\newacronym[description={\glslink{apig}{Application Program Interface}}]
    {api}{API}{Application Program Interface}

\newacronym[description={\glslink{umlg}{Unified Modeling Language}}]
    {uml}{UML}{Unified Modeling Language}

%**************************************************************
% Glossario
%**************************************************************
%\renewcommand{\glossaryname}{Glossario}

\newglossaryentry{apig}
{
    name=\glslink{api}{API},
    text=Application Program Interface,
    sort=api,
    description={in informatica con il termine \emph{Application Programming Interface API} (ing. interfaccia di programmazione di un'applicazione) si indica ogni insieme di procedure disponibili al programmatore, di solito raggruppate a formare un set di strumenti specifici per l'espletamento di un determinato compito all'interno di un certo programma. La finalità è ottenere un'astrazione, di solito tra l'hardware e il programmatore o tra software a basso e quello ad alto livello semplificando così il lavoro di programmazione}
}

\newglossaryentry{cluster}
{
    name=\glslink{cluster}{cluster},
    text=cluster,
    sort=cluster,
    description={(indicato anche come computer cluster) insieme di macchine connesse tra loro che lavorano in parallelo.
    L'utilizzo di questi sistemi permette di distribuire un'elaborazione molto complessa tra le varie macchine, aumentando la potenza di calcolo del sistema e/o garantendo una maggiore disponibilità di servizio, a prezzo di un maggior costo e complessità di gestione dell'infrastruttura: per essere risolto, il problema che richiede molte elaborazioni, viene infatti scomposto in sottoproblemi separati i quali vengono risolti ciascuno in parallelo su tutti i nodi che compongono il cluster.}
}

\newglossaryentry{web app}
{
	name=\glslink{web app}{web app},
	text=web app,
	sort=web app,
	description={sistema di tipo client-server in cui l'interfaccia utente e la logica client-side viene eseguita in un browser web.}
}

\newglossaryentry{Software Development Lifecycle (SDLC)}
{
	name=\glslink{Software Development Lifecycle}{SDLC},
	text=Software Development Lifecycle,
	sort=SDLC,
	description={processo di divisione del lavoro di sviluppo software in fasi distinte per migliorare la progettazione, la gestione del prodotto e la gestione del progetto.}
}