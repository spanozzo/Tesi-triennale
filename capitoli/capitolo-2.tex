% !TEX encoding = UTF-8
% !TEX TS-program = pdflatex
% !TEX root = ../tesi.tex

%**************************************************************
\chapter{Processi e metodologie}
\label{cap:processi-metodologie}
%**************************************************************

%**************************************************************
\section{Aspettative personali}
La mia esperienza lavorativa si è sempre limitata a piccoli lavori occasionali, un periodo in cui ho svolto un progetto esterno per un'azienda ed in seguito uno stage nella stessa azienda, entrambi durante la scuola secondaria di secondo grado.\\
Già da queste piccole esperienze, avevo capito quanto la pratica sul campo fosse un'ottima metodologia per applicare attivamente quanto studiato durante la carriera da studente ed, inoltre, apprendere nuove nozioni. Anche negli anni universitari, mi è spesso capitato di ritrovare nella teoria ciò che già conoscevo nella pratica, rendendo di fatto il mio studio più facilitato ma anche più completo rispetto alle conoscenze che avevo, che risultavano lacunose ad un'analisi a posteriori. Per questo motivo, anche dall'attività di stage le mie speranze erano quelle di poter apprendere nuove nozioni tramite un approccio che fosse più pratico di quello che normalmente viene utilizzato all'interno dell'ambiente universitario.
 
Per far sì che le mie opzioni di scelta fossero quanto più numerose possibili, ho deciso di partecipare all'iniziativa denominata Stage-IT\footcite{http://informatica.math.unipd.it/laurea/stageit.html} organizzata dall'Università di Padova. Prima di andare all'evento, ho stilato una lista di possibili aziende a cui presentarmi, facendo soprattutto attenzione ai progetti proposti. Durante l'evento ho quindi colto l'occasione per discutere dell'attività di stage con molteplici aziende. Molte di loro operavano in campi che a mio parere non risultavano essere particolarmente interessanti, ma qualche azienda ha colto la mia curiosità e mi ha permesso di avere con loro un colloquio di presentazione reciproca. Alcune di queste erano più interessate a stage per inserimento lavorativo e di durata di circa 6 mesi, che a uno finalizzato per la tesi, e quindi ho dovuto scartarle a priori nonostante il mio interesse. \\
Fortunatamente, l'azienda I.T. Euro Consulting proponeva uno stage adeguato per lo sviluppo della tesi e gli argomenti coinvolti erano pienamente di mio interesse: durante il colloquio conoscitivo comunque non mi è stato subito presentato un progetto di stage, ma piuttosto mi è stata esposta la struttura aziendale, la focalizzazione dello stage nell'ambito \textins{big data} e la propensione dell'azienda per l'innovazione e l'interesse a conoscere nuovi punti di vista come quello di uno studente universitario. Tutto ciò mi ha attratto fin da subito e quindi, dopo un paio di incontri successivi avvenuti in azienda, si è proceduto alla presentazione del progetto di stage vero e proprio.\\\\
I miei obiettivi iniziali per un progetto di stage erano i seguenti:
\begin{itemize}
	\item Entrare in contatto con nuove tecnologie;
	\item Vedere e capire il funzionamento di una realtà aziendale in ambito tecnologico ed ICT;
	\item Poter lavorare con il supporto di personale qualificato;
	\item Vedere in pratica come le nozioni apprese in aula o da studio personale vengono realmente applicate nel mondo lavorativo.
\end{itemize}
Considerando i miei obiettivi e la lista di aziende con le quali ho avuto un contatto a Stage-IT, ho deciso di svolgere il progetto di IT Euro Consulting, in quanto di maggiore interesse rispetto ai progetti offerti da altre aziende.
Successivamente al mio impegno con l'azienda, la mia curiosità verso l'argomento di stage si è intensificata. Le aspettative che maggiormente sentivo erano:
\begin{itemize}
	\item Mettersi in gioco in un'azienda con partner di un certo livello nel mio campo di studi;
	\item Instaurare con il personale discussioni su esperienze e punti di vista diversi sulle varie tecnologie;
	\item Entrare in contatto con tecnologie nuove e sempre più di largo utilizzo;
	\item Conoscere il funzionamento di uno strumento come Hadoop e tutti i tool inerenti;
	\item Apprendere come effettuare un'analisi su un insieme di dati distribuiti in un \gls{cluster};
	\item Imparare come progettare e quali sono le \textit{best practies} per realizzare una \gls{web app} utilizzando Java EE.
\end{itemize}

%**************************************************************
\section{Aspettative aziendali}
Quest'anno IT Euro Consulting ha deciso, per la prima volta, di attivare un progetto di stage universitario che non avesse come fine ultimo l'assunzione del tirocinante. I motivi di questa scelta sono molteplici e condivisibili. \\
In primo luogo è necessario distinguere i percorsi che l'azienda ha deciso di far intraprendere ai diversi tirocinanti in base allo scopo dello stage. \\ 
Per gli studenti, appena laureati o laureandi, il cui fine è l'assunzione al termine del tirocinio è previsto un periodo in azienda di circa 6 mesi, durante il quale, dopo il primo periodo formativo, si viene inseriti in progetti già avviati e quindi affiancati dal team a cui è assegnato il lavoro. Negli ultimi 3 anni infatti i reparti sviluppo e \textit{big data} si sono molto ampliati e sono state assunte dall'azienda molteplici persone in seguito ad uno stage o tirocinio. Oggigiorno infatti la maggior parte del personale appartenente ai reparti sviluppo e \textit{big data} è qualificata con un titolo di laurea in Informatica o Ingegneria. \\
Nell'altro caso, ovvero lo stage curricolare, come di mio interesse, la durata è di circa 300-350 ore e si distingue in una fase di apprendimento ed in una fase di progetto, sempre affiancati da uno o più tutor interni. Durante il progetto, si concede al tirocinante maggiore libertà sulla scelta delle tecnologie da utilizzare e su decisioni progettuali. Queste poi vengono discusse con il tutor in modo da correggere eventuali scelte errate frutto dell'inesperienza del tirocinante. \\
I vantaggi di questi due percorsi sono in parte comuni: in primo luogo si ha l'inserimento in organico di nuove risorse provenienti dal mondo universitario. Assumere anche solo provvisoriamente una figura per uno stage proveniente dal mondo universitario giova all'azienda in quanto il personale ha la possibilità di confrontarsi ed aggiornarsi con costui sui nuovi insegnamenti e corsi universitari. Questa vicinanza è però anche utile per lo studente, in quando ha la possibilità di vedere concretamente come quanto appreso in aula sia implementato effettivamente nel mondo reale e di capire come funzioni realmente un'azienda, se non ha già avuto esperienze simili durante la sua carriera.
Il secondo motivo è la possibilità per l'azienda di comprendere il livello di preparazione medio degli studenti universitari ed essere attivi nello scrutare quelli più meritevoli che possono portare ad un vantaggio competitivo considerevole.
Per quanto riguarda il mio percorso di stage, quello curricolare, si possono riconoscere altri due vantaggi: entrambi derivano dalla maggior libertà che si concede allo studente per risolvere problemi che solitamente in azienda si risolverebbero tramite soluzioni già consolidate. La possibilità di sfruttare nuove tecnologie senza la necessità di perdita di tempo del personale aziendale, permette di rimanere aggiornati sul continuo rilascio di librerie e \textit{framework} che potrebbero portare benefici ai vari team in termini di efficienza, sia sul tempo di sviluppo del software, sia per quanto riguarda le performance del prodotto stesso.
La maggiore libertà sulla progettazione e lo sviluppo concessa allo stagista consente anche di valutare vantaggi e svantaggi di soluzioni architetturali che si erano scartate prematuramente o che non erano state considerate a priori.\\

Tutto ciò mi ha portato prima dell'inizio dell'attività di stage, a concordare insieme
al tutor aziendale i requisiti del progetto, che in seguito pero sono stati modicati, con
un paio di aggiornamenti del Piano di Lavoro. Ne viene riportata la versione nale.
Gli obiettivi sono divisi in tre categorie: obiettivi obbligatori, obiettivi desiderabili e
obiettivi facoltativi.

%**************************************************************
\section{Presentazione del progetto}


%**************************************************************
\section{Vincoli}

