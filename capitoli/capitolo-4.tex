% !TEX encoding = UTF-8
% !TEX TS-program = pdflatex
% !TEX root = ../tesi.tex

%**************************************************************
\chapter{Valutazione retrospettiva}
\label{cap:valutazione-retrospettiva}
%**************************************************************

\section{Soddisfacimento degli obiettivi}
Prima dell'inizio dell'attività di stage, io e il mio tutor abbiamo redatto il Piano di Lavoro. All'interno di esso, oltre ad una sintesi dell'organizzazione del tirocinio e ad una serie di ulteriori informazioni, abbiamo fissato anche gli obiettivi obbligatori e desiderabili che avrei dovuto raggiungere nell'arco delle 320 ore di stage, visibili nella tabella \hyperref[obiettivi_stage]{2.1} del secondo capitolo.\\
Gli obiettivi desiderabili sono tutti degli incrementi rispetto a quelli obbligatori. In questo modo mi sarei potuto concentrare su quelli più importanti per primi e solo in seguito, una volta finiti, approfondire gli altri.\\ 
Sono riuscito a raggiungere tutti gli obiettivi obbligatori e parzialmente quelli desiderabili. Di quest'ultimi, infatti, ho solo in parte soddisfatto il seguente:
\begin{itemize}
	\item Sviluppo di un'applicazione Java EE \textit{3-tier}, composta da un'interfaccia grafica HTML5/Angular in grado di visualizzare i dati dei risultati in modalità grafica e un Web Service RESTful per la presentazione dei dati in formato JSON recuperati con Hive/Impala.
\end{itemize}
Questo perchè, come precedentemente già descritto, l'amministratore di sistema dell'azienda non ha potuto configurare le \gls{Java JDBC} in tempo per il mio utilizzo e, quindi, ho recuperato i dati ricavati dall'\hyperref[dataset]{elaborazione del dataset} da tabelle \gls{CSV} salvate sulla macchina locale invece che sul \gls{cluster}, che sarebbero state recuperate utilizzando Hive o Impala. Comunque ho raggiunto le rimanenti specifiche richieste dell'obiettivo. \\
Durante il tempo guadagnato dalla mancanza della necessità di implementare la connessione del sistema con il \gls{cluster}, ho ottimizzato il prodotto, soprattutto per la visualizzazione dei risultati mediante gli script Angular sviluppati. \\\\
Il bilancio del soddisfacimento degli \hyperref[obiettivi_stage]{obiettivi di stage} è di seguito riportato.
\begin{table}[!h] %
	\caption{Bilancio soddisfacimento obiettivi di stage}
	\label{tab:bilancio-app}
	\centering
	\begin{tabular}{ c | c | c | c}
		\textbf{Importanza} & \textbf{Individuati} & \textbf{Soddisfatti} & \textbf{\%}\\
		\hline
		\hline
		\\[-2.5mm]
		Obbligatori & 4 & 4 & 100 \\
		\hline
		\\[-2.5mm]
		Desiderabili & 3 & 2 & 67 \\
		\hline
	\end{tabular}
\end{table}%

\section{Conoscenze acquisite}
Le conoscenze che ho aggiunto o ampliato rispetto al bagaglio formativo in mio possesso prima dell'inizio dello stage sono molteplici.

\paragraph{Lavoro individuale e confronto}
Durante la maggior parte dell'attività di stage, visti gli obiettivi di studio di nuove tecnologie, analisi e progettazione, ho dovuto lavorare da solo. Questo aspetto non mi è nuovo, ma applicarlo in un ambiente lavorativo comune mi ha fatto raggiungere un grado di autonomia maggiore, rendendomi contemporaneamente consapevole di quanto sia importante confrontarsi con gli altri durante queste attività. \\
In particolare, nei momenti di confronto con il personale aziendale, ho notato come io non abbia tenuto in considerazione certi aspetti secondari di sviluppo, ma mi sia concentrato inizialmente solo sulla mia idea. Questo aspetto, se sottovalutato e non tenuto sotto controllo, mi avrebbe potuto fare perdere tempo inutilmente in fasi avanzate del progetto.\\
Tale esperienza mi ha portato dunque a rivedere il mio stile di lavoro, tipicamente molto autonomo, imparando ad interagire con le altre personalità aziendali e a confrontarmi con esse. Oltre a questo, mi ha permesso di capire quanto siano importanti le opinioni e i pareri di tutte le persone che partecipano a un progetto e quanto, confrontandosi nei momenti di necessità, sia possibile raggiungere in tempi più rapidi a una soluzione. \\
Tale insegnamento porterà sicuramente a cambiare il mio modo di interagire anche con i miei futuri colleghi universitari durante lo svolgimento di progetti, sia all'interno dei corsi universitari che al di fuori di essi.

\paragraph{Big Data}
Grazie a questo stage ho potuto constatare con i miei occhi come vengono mantenuti in persistenza e analizzati grandi quantità di dati che possono portare benefici, sia per i privati sia per il pubblico. \\
Entrare in contatto con realtà che si occupano di questo lavoro e alle modalità con cui questo viene realizzato, mi ha permesso di focalizzare la mia attenzione sull'importanza dei dati che creiamo ogni giorno. \\
Avere quindi la possibilità di studiare gli strumenti del mondo \textit{big data} e lavorare con essi, mi ha permesso di comprendere meglio i processi che avvengono dalla creazione del dato fino alla sua elaborazione. Inoltre, seppur non oggetto di questo stage, sono riuscito a capire anche l'importanza dell'analisi finale a cui è sottoposto il dato, con la finalità di produrre un risultato oggettivo su cui basare le strategie aziendali.  

\paragraph{Tecnologie}
Durante questo periodo di tirocinio, ho avuto la possibilità di approcciarmi a tecnologie per me nuove, sia inerenti allo sviluppo software (Java EE e Angular), sia per quanto riguarda il mondo \textit{big data} (Hadoop e i suoi tool). \\
Ho avuto come riferimento in azienda, oltre che al tutor, alcuni sviluppatori e analisti a cui rivolgermi per dubbi o problemi. In questo modo ho potuto apprendere, confrontandomi direttamente con loro, le \textit{best practice} da utilizzare.

\paragraph{Presentazione ed esposizione verbale}
Il confronto continuo con i tutor aziendali e il personale che mi circondava, mi ha permesso di farmi maturare nella produzione ed esposizione verbale dei risultati che ho ottenuto. Infatti, ho compreso quanto risulti utile esporre ad altre persone i propri dubbi e il lavoro svolto, in quanto ho maturato uno spirito più critico verso ciò che facevo, sia in termini di attività svolte che di risultati finali.

\paragraph{L'azienda}
Durante il periodo in azienda ho avuto l'occasione di vedere più da vicino come è strutturata questo tipo di realtà e il suo funzionamento interno in termini di processi aziendali. Ho realizzato l'importanza della suddivisione e la distinzione dei compiti fra i vari componenti aziendali, nonostante sia fondamentale la comunicazione e il contatto fra i vari ruoli per avere un confronto critico imparziale.

\section{Valutazione personale}
Non avendo esperienze lavorative rilevanti maturate, se non per progetti individuali o effettuati durante il periodo precedente l'università, visto che in questi anni ho preferito concentrami sugli studi e avere la possibilità di frequentare le lezioni dei corsi universitari, durante questo stage ho avuto modo di farmi una prima opinione sul gap formativo che sussiste fra università e lavoro. \\ 
Ritengo che l'università sia il luogo migliore per poter apprendere le basi di ciò che poi si dovrà applicare alla vita lavorativa, oltre che un'occasione unica per confrontarsi con colleghi e persone più esperte riguardo argomenti di interesse, per maturare e sviluppare abilità per risolvere i problemi in modo differente da quanto solitamente fatto. \\ 
Dopo la scuola secondaria di secondo grado ho scelto di proseguire gli studi iscrivendomi al corso universitario di Informatica per avere una migliore preparazione prima di entrare nel mondo del lavoro. \\
Durante i primi anni, ho trovato molti corsi non molto stimolanti, soprattutto perché non riuscivo ad applicare i concetti studiati a un vero e proprio ambito lavorativo che catturasse veramente la mia attenzione. \\ 
Quest'ultimo anno però, fortunatamente, ho rivalutato la mia opinione. Ho potuto apprezzare questo corso di laurea e sviluppare conoscenze utili che mi interessano veramente, così da poter capire su quali argomenti volermi concentrare. \\
Durante quest'ultimo anno il carico di lavoro si è distinto notevolmente rispetto ai primi: ci sono un paio di progetti di gruppo che impegnano per un numero di ore non indifferente, che portano ad assumersi responsabilità molto differenti da quanto sperimentato prima. Non esiste, dunque, un avvicinamento graduale a questi impegni e, oltre a ciò, gli strumenti e i concetti affrontati sono molteplici e concentrati in un periodo di tempo limitato che può portare inizialmente a delle difficoltà. \\
Ritengo che potrebbe essere utile introdurre queste conoscenze gradualmente insieme alla semplice programmazione fine a sè stessa, così che venga naturale svolgere attività come il versionamento del codice o i processi di supporto ad un progetto al momento dello sviluppo di lavori più complessi. \\\\
\noindent Per quanto riguarda la mia personale opinione su questa esperienza, posso dire che mi ritengo in buona parte appagato. Se infatti da una parte ho prodotto pienamente quanto richiesto dall'azienda, che si è dimostrata soddisfatta, dall'altra devo ammettere che, seppur molto interessanti, gli argomenti trattati sono stati leggermente slegati dagli insegnamenti affrontati durante il percorso di studio. 
Ritengo infatti che uno stage, i cui obiettivi principali riguardano il mondo \textit{big data}, necessiti di un'insegnamento, o un'integrazione in uno già esistente, che permettano a studenti e futuri tirocinanti di comprendere l'importanza dello studio dei dati prodotti sempre in maggior misura. Se infatti in corsi relativi alle tecnologie di internet si incrementasse questo tipo di studi, penso che ogni studente ne trarrebbe dei benefici, così da considerare questo aspetto in futuri progetti.
Oltre a ciò, considero che uno studio più approfondito di alcuni modelli statistici di base, come fornito dal corso "\textit{Data Mining}", possa rivelarsi utile per uno studente, non solo nell'ambito di questo stage.
Considero inoltre che un progetto più complesso, contenente anche lo sviluppo di un modello statistico per i dati elaborati, possa rendere più completa l'esperienza di tirocinio e dare più soddisfazioni allo studente. Questo, però, essendo un'attività impegnativa, richiederebbe molte più ore di quante richieste per lo stage curricolare e quindi non sarebbe indicata come esperienza al termine della laurea triennale. \\
Tutto sommato, comunque, ritengo di aver acquisito durante questo percorso delle conoscenze molto utili e complete per affrontare nuovamente altri problemi similari in modo più maturo e con una visione differente rispetto a quanto avrei fatto in precedenza.