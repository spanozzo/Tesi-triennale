% !TEX encoding = UTF-8
% !TEX TS-program = pdflatex
% !TEX root = ../tesi.tex

%**************************************************************
% Sommario
%**************************************************************
\cleardoublepage
\phantomsection
\pdfbookmark{Sommario}{Sommario}
\begingroup
\let\clearpage\relax
\let\cleardoublepage\relax
\let\cleardoublepage\relax

\chapter*{Sommario}

Il presente documento descrive il lavoro svolto durante il periodo di stage del laureando Stefano Panozzo presso l'azienda I.T. Euro Consulting S.r.l. di Padova. Lo stage è stato svolto alla conclusione del percorso di studi della Laurea Triennale ed è durato in totale 320 ore.
Gli obiettivi da raggiungere erano molteplici.\\
La prima richiesta dell'azienda era analizzare la struttura del \gls{cluster} in cui risiedevano i dati utilizzati in seguito. 
Successivamente era richiesta l'analisi e la trasformazione del dataset di interesse per estrarre ed ottenere nuove informazioni utili per creare un modello che prevedesse il target desiderato. 
Infine, era richiesta la progettazione e lo sviluppo di una \gls{web app} per la rappresentazione dei risultati ottenuti in precedenza.
I primi due capitoli del presente documento hanno lo scopo di presentare il contesto aziendale in cui è stato sostenuto lo stage e di spiegare come il progetto di stage si renda utile all’interno della strategia aziendale. Il terzo capitolo documenta lo svolgimento dello stage descrivendo le attività che sono state portate a termine, i punti salienti del progetto stesso e le principali scelte progettuali. Il quarto ed ultimo capitolo presenta infine una valutazione dello svolgimento dello stage rispetto agli obiettivi aziendali e alle conoscenze acquisite dallo studente.

%\vfill
%
%\selectlanguage{english}
%\pdfbookmark{Abstract}{Abstract}
%\chapter*{Abstract}
%
%\selectlanguage{italian}

\endgroup			

\vfill

