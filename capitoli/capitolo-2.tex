% !TEX encoding = UTF-8
% !TEX TS-program = pdflatex
% !TEX root = ../tesi.tex

%**************************************************************
\chapter{Processi e metodologie}
\label{cap:processi-metodologie}
%**************************************************************

%**************************************************************
\section{Aspettative personali}
La mia esperienza lavorativa si è sempre limitata a piccoli lavori occasionali, un periodo in cui ho svolto un progetto esterno per un'azienda ed in seguito uno stage nella stessa azienda, entrambi durante la scuola secondaria di secondo grado.\\
Già da queste piccole esperienze, avevo capito quanto la pratica sul campo fosse un'ottima metodologia per applicare attivamente quanto studiato durante la carriera da studente ed, inoltre, apprendere nuove nozioni. Anche negli anni universitari, mi è spesso capitato di ritrovare nella teoria ciò che già conoscevo nella pratica, rendendo di fatto il mio studio più facilitato ma anche più completo rispetto alle conoscenze che avevo, che risultavano lacunose ad un'analisi a posteriori. Per questo motivo, anche dall'attività di stage le mie speranze erano quelle di poter apprendere nuove nozioni tramite un approccio che fosse più pratico di quello che normalmente viene utilizzato all'interno dell'ambiente universitario.
 
Per far sì che le mie opzioni di scelta fossero quanto più numerose possibili, ho deciso di partecipare all'iniziativa denominata Stage-IT\footcite{http://informatica.math.unipd.it/laurea/stageit.html} organizzata dall'Università di Padova. Prima di andare all'evento, ho stilato una lista di possibili aziende a cui presentarmi, facendo soprattutto attenzione ai progetti proposti. Durante l'evento ho quindi colto l'occasione per discutere dell'attività di stage con molteplici aziende. Molte di loro operavano in campi che a mio parere non risultavano essere particolarmente interessanti, ma qualche azienda ha colto la mia curiosità e mi ha permesso di avere con loro un colloquio di presentazione reciproca. Alcune di queste erano più interessate a stage per inserimento lavorativo e di durata di circa 6 mesi, che a uno finalizzato per la tesi, e quindi ho dovuto scartarle a priori nonostante il mio interesse. \\
Fortunatamente, l'azienda I.T. Euro Consulting proponeva uno stage adeguato per lo sviluppo della tesi e gli argomenti coinvolti erano pienamente di mio interesse: durante il colloquio conoscitivo comunque non mi è stato subito presentato un progetto di stage, ma piuttosto mi è stata esposta la struttura aziendale e la focalizzazione dello stage nell'ambito \textins{big data} e la propensione dell'azienda per l'innovazione e per nuovi punti di vista come quello di uno studente universitario. Tutto ciò mi ha attratto molto fin da subito e quindi, dopo un paio di incontri successivi avvenuti in azienda, si è proceduto alla presentazione del progetto di stage vero e proprio.

%**************************************************************
\section{Aspettative aziendali}

%**************************************************************
\section{Presentazione del progetto}

%**************************************************************
\section{Vincoli}