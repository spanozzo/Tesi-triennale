% !TEX encoding = UTF-8
% !TEX TS-program = pdflatex
% !TEX root = ../tesi.tex

%**************************************************************
\chapter{Valutazione retrospettiva}
\label{cap:valutazione-retrospettiva}
%**************************************************************

\section{Soddisfacimento degli obiettivi}
Prima dell'inizio dell'attività di stage, io e il mio tutor abbiamo redatto il Piano di Lavoro. All'interno di esso, oltre ad una sintesi dell'organizzazione del tirocinio e ad una serie di ulteriori informazioni, abbiamo fissato anche gli obiettivi obbligatori e desiderabili che avrei dovuto raggiungere nell'arco delle 320 ore di stage, visibili in \hyperref[obiettivi_stage]{questa} tabella.\\
Gli obiettivi desiderabili che abbiamo fissato erano tutti degli incrementi rispetto a quelli obbligatori. In questo modo mi sarei potuto concentrare su quelli più importanti per primi e solo in seguito, una volta finiti, approfondire gli altri.\\ 
Sono riuscito a raggiungere tutti gli obiettivi obbligatori e parzialmente quelli desiderabili. Di quest'ultimi, infatti ho solo in parte soddisfatto il seguente:
\begin{itemize}
	\item Sviluppo di un'applicazione Java EE \textit{3-tier}, composta da un'interfaccia grafica HTML5/Angular in grado di visualizzare i dati dei risultati in modalità grafica e un Web Service RESTful per la presentazione dei dati in formato JSON recuperati con Hive/Impala.
\end{itemize}
Questo perchè, come precedentemente già descritto, l'amministratore di sistema dell'azienda non ha potuto configurare le \gls{Java JDBC} in tempo per il mio utilizzo e, quindi, ho recuperato i dati ricavati dall'\hyperref[dataset]{elaborazione del dataset} da tabelle \gls{CSV} salvate sulla macchina locale invece che sul \gls{cluster} e poi recuperate utilizzando Hive o Impala. Inoltre, ho raggiunto le specifiche richieste dal rimanente obiettivo. \\
Il tempo guadagnato dalla mancanza della necessità di implementare le connessioni del sistema con il \gls{cluster}, è stato impiegato per ottimizzare il prodotto, soprattutto per la visualizzazione dei risultati mediante gli script Angular sviluppati.

\section{Conoscenze acquisite}
Le conoscenze che sento di aver aggiunto o ampliato rispetto al bagaglio formativo in mio possesso prima dell'inizio dello stage sono molteplici. Esse sono riassunte di seguito.

\paragraph{Lavoro individuale}
Durante la maggior parte dell'attività di stage, visti suoi obiettivi di studio di nuove tecnologie, analisi e progettazione, ho dovuto lavorare da solo. Questo aspetto non mi è nuovo, ma applicarlo in un ambiente lavorativo comune mi ha fatto raggiungere un grado di autonomia maggiore, rendendomi contemporaneamente consapevole di quanto sia importante confrontarsi con gli altri durante queste attività. \\
In particolare, nei momenti di confronto con il resto del personale aziendale, ho notato come, alcune volte, io non abbia tenuto in considerazione certi aspetti secondari di sviluppo, ma mi sia concentrato inizialmente sulla mia idea. Questo aspetto, se sottovalutato e non tenuto sotto controllo, mi avrebbe potuto fare perdere tempo inutilmente in fasi avanzate del progetto.\\
Tale esperienza mi ha portato dunque a rivedere il mio stile di lavoro, tipicamente molto autonomo, imparando ad interagire con le altre personalità aziendali e a confrontarmi con esse. Oltre a questo, mi ha permesso di capire quanto siano importanti le opinioni ed i pareri di tutte le persone che partecipano ad un progetto e quanto, confrontandosi nei momenti di necessità, sia possibile raggiungere in tempi più rapidi ad una soluzione. \\
Tale insegnamento porterà sicuramente a cambiare il mio modo di interagire anche con i miei futuri colleghi universitari durante lo svolgimento di progetti, sia all'interno dei corsi universitari che al di fuori di essi.
\paragraph{}
\paragraph{Presentazioni ed esposizione verbale}

\section{Valutazione personale}