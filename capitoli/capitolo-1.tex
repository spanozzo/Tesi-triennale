% !TEX encoding = UTF-8
% !TEX TS-program = pdflatex
% !TEX root = ../tesi.tex

%**************************************************************
\chapter{Il contesto aziendale}
\label{cap:introduzione}
%**************************************************************

\noindent Esempio di utilizzo di un termine nel glossario \\
\gls{api}. \\

\noindent Esempio di citazione in linea \\
\cite{site:agile-manifesto}. \\

%**************************************************************
\section{Il profilo aziendale}

I.T. Euro Consulting S.r.l.\footcite{https://www.itecons.it} è un'azienda di medie dimensioni con sede legale a Padova, nata nel 2007 e facente parte del gruppo SCAI, presente su tutto il territorio italiano. 
Dalla sua nascita si è sempre occupata prevalentemente di consulenza, \textit{System Integration} ed \textit{Application Management}, in ambito ICT, operando in tutti i principali settori di mercato: bancario ed assicurativo, industria, pubblica amministrazione e servizi.
Nel corso degli anni l'azienda ha consolidato le proprie conoscenze soprattutto nei seguenti ambiti, offrendo svariati servizi:
\begin{itemize}
	\item \textbf{Big Data}: supporto alle aziende nel loro processo di crescita e cambiamento, tramite moderne soluzioni di \textit{Business Intelligence} e la possibilità di prevedere scenari ed eventi futuri e prendere le più opportune decisioni operative o di business grazie all'analisi della gran mole di dati che ogni giorno vengono creati. Vengono quindi offerti servizi di big data engineer, big data scientist, big data architect e big data administrator;
	\item \textbf{Internet of Things}: soluzioni end to end, basate su tecnologie leader di mercato che consentono di indirizzare in modo efficace la realizzazione di sistemi Iot accelerando la realizzazione di componenti web e mobile per la raccolta, la visualizzazione e l’analisi dei dati;
	\item \textbf{Reference Architecture}: intesa come best practice e struttura di base per un insieme di domini applicativi all’interno di un’organizzazione, la quale agevola il continuo allineamento dei processi e delle strategie con le giuste soluzioni tecnologiche. Vengono quindi offerti servizi di assessment, design e consulenza;
	\item \textbf{DevOps}: automatizzazione delle attività manuali nelle diverse fasi del \gls{Software Development Lifecycle (SDLC)}. Il modello DevOps non si concentra esclusivamente sull’ introduzione di nuovi tool, ma è inteso come una combinazione di cultura, processi unita agli strumenti di automazione. Vengono quindi offerti servizi di assessment e consulenza; 
	\item \textbf{System Integration}: servizi di consulenza o interventi progettuali per aiutare le aziende a gestire al meglio le proprie strutture tecnologiche complesse e soluzioni applicative per semplificare la coesione fra i vari sottosistemi che compongono la struttura;
	\item \textbf{Application Management}: servizi di manutenzione correttiva, adattativa ed evolutiva di soluzioni applicative durante il loro intero ciclo di vita;
	\item \textbf{Customer Relationship Management}: con l'obiettivo di ottenere una visione completa per perseguire uno scenario di \textit{Single Customer View}, abilitante al dialogo one-to-one tra l'organizzazione ed il proprio cliente indipendentemente dalle canalità attraverso le quali avviene l'interazione;
	\item \textbf{System \& Data Administration}: servizio consultivo svolto avvalendosi di un insieme di strategie, processi e regole che consentono di gestire i sistemi e trattare i dati fondamentali per lo sviluppo aziendale. Vengono quindi offerti servizi di database administration, database security, data governance, data analysis e scheduling management.
\end{itemize}

%**************************************************************
\section{Dominio applicativo}

Introduzione all'idea dello stage.

%**************************************************************
\section{Tecnologie utilizzate}

%**************************************************************
\section{Processi aziendali}

%**************************************************************
\section{Tipo di clientela}

%**************************************************************
%\section{Propensione dell'azienda per l'innovazione}

%**************************************************************